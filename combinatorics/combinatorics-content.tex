\mathchardef\mh="2D

\begin{prb}[2010 AIME II-10]
Find the number of second degree polynomials $f(x)$ with integer coefficients
and integer zeroes for which $f(0) = 2010$.
\end{prb}

\ifsolutions
\begin{proof}[Solution]
Let the roots be $r, s$ such that $f(x) = a(x - r)(x - s)$. We will work with
the group $S_2$ consisting of two elements, identity and swapping the roots. We
first factor $2010 = 2 \cdot 3 \cdot 5 \cdot 67$. We count the number of ordered
triples $(a, r, s)$ by first assigning every copy of each prime to one of the
three integers, and then picking signs for the first two. This yields $4 \cdot
3^4 = 324$ possible combinations. Notice that $r$ and $s$ can never be equal
unless $r = s = \pm 1$, which amount to $2$ solutions. By Burnside's Lemma, the
number of polynomials is then
\[ \frac{1}{2} \cdot (324 + 2) = \boxed{163}. \]
\end{proof}
\fi

\begin{prb}[1996 AIME]
Two squares of a $7 \times 7$ checkerboard are painted yellow and the rest
green. Two paintings are equivalent if one can be obtained by applying a
rotation to the other. How many inequivalent color schemes are there?
\end{prb}

\ifsolutions
\begin{proof}[Solution]
There are $\binom{49}{2}$ ways to color two squares yellow. We will apply
Burnside's Lemma on the additive group of integers modulo $4$. We can think of
each number as the number of $90 \degree$ rotations made to the board.

$0$ trivially fixes all possible colorings.

$1$ and $3$ fix no colorings after careful examination. This can be determined
by noting that if a tile is colored off-center, it rotates through $4$ different
positions, so there must be at least $4$ yellow tiles. There is only one center
tile so this must be the case.

$2$ fixes all colorings where two tiles are colored in radially opposite
positions. We cannot color the center, so there are $\frac{49 - 1}{2} = 24$ such
colorings.

By Burnside's Lemma, there are $\frac{1176 + 0 + 24 + 0}{4} = \boxed{300}$
unique boards.
\end{proof}
\fi

\begin{prb}
Let $N$ be the number of subsets $\lbrace x, y, z, t \rbrace \subset \NN$
satisfy $17 \leq x < y < z < t$ and $x + y + z + t = 2018$. Compute the last $3$
digits of $24N$.
\end{prb}

\ifsolutions
\begin{proof}[Solution]
Rewrite $x, y, z, t$ as
\[ \begin{aligned}
x &= 17 + a \\
y &= 18 + b \\
z &= 19 + c \\
t &= 20 + d. \\
\end{aligned} \]
The original equation then becomes $a + b + c + d = 1944$ with $a, b, c, d \in
\NN_0$. We wish to count the number of such solutions such that $a \leq b \leq c
\leq d$. Let $X$ be the set of all solutions, and we will let $S_4$ act on this
set. We classify all $24$ operations of $S_4$ as follows.

\begin{enumerate}
\item
The identity permutation. There is $1$ of these. This fixes all $\binom{1944 +
3}{3}$ solutions.
\item
Transpositions ($i \mapsto j \mapsto i$, denoted $(i \; j)$, everything else is
fixed). There are $1! \cdot \binom{4}{2} = 6$ of these. The number of solutions
where two values are equal is $1945 + 1943 + 1941 + \cdots + 1 = 973^2$, found
by taking $a = b = 0, a = b = 1, \dots$.
\item
Two disjoint transpositions (e.g. $(1 \; 2) (3 \; 4)$). There are $1! \cdot 1!
\cdot \frac{1}{2} \binom{4}{2} \binom{2}{2} = 3$ of these. Each one of these has
solutions of the form $2a + 2b = 1944$, or $a + b = 972$, so there are $973$
such solutions.
\item
A permutation of $3$ elements. There are $2! \cdot \binom{4}{3} = 8$ of these.
Each of these has $1 + \left\lfloor \frac{1944}{3} \right\rfloor = 649$
solutions.
\item
A permutation of $4$ elements. There are $3! \cdot \binom{4}{4} = 6$ of these.
$1944$ is a multiple of $4$, so there is indeed a solution where all $4$ values
are equal.
\end{enumerate}
Using Burnside's Lemma, we get
\[ N = \frac{\binom{1947}{3} + 6 \cdot 973^2 + 3 \cdot 973 + 8 \cdot 649 +
6}{24}. \]
Thus
\[ \begin{aligned}
24N \pmod{1000} &\equiv 649 \cdot 973 \cdot 945 + 6 \cdot 27^2 - 3 \cdot 27 + 8
\cdot 649 + 6 \\
&\equiv 765 + 374 - 81 + 192 + 6 \\
&\equiv \boxed{256}. \\
\end{aligned} \]
\end{proof}
\fi

\begin{prb}[2010 AMC 12A-25]
Two quadrilaterals are considered equivalent if one can be obtained by applying
a rotation and or a translation to the other. How many different convex cyclic
quadrilaterals are there with integer side lengths and perimeter $32$?
\end{prb}

\ifsolutions
\begin{proof}[Solution]
We first note that any quadrilateral can be made into a cyclic one by changing
the angles.

Let $a, b, c, d$ be the side lengths (not necessarily in order) of a cyclic
quadrilateral with $a + b + c + d = 32$. We consider the group of $4$ side
permutations after we have assigned each side its length. This is essentially
``rotating $90 \degree$.'' Notice that we do not need to consider swapping
sides (given by $S_4$), as it does not yield a rotation or translation.

Doing nothing fixes all cyclic quadrilaterals. Assume WLOG $a \leq b \leq c \leq
d$. We require $d \leq 15$ for triangle inequality reasons. There are
$\binom{31}{3}$ ways of partitioning $32$, and we will subtract out the cases
for $d > 15$. Namely, there are $\binom{31 - d}{2}$ ways of partitioning $a + b
+ c = 32 - d$. There are $\binom{15}{2} + \binom{14}{2} + \cdots + \binom{2}{2}
= \binom{16}{3}$ such partitions. We then subtract $4$ times this number because
$a$ can be any of the four side lengths. The number of cyclic quadrilaterals is
$\binom{31}{3} - 4 \cdot \binom{16}{3} = 4495 - 2240 = 2255$.

Rotating once only fixes squares. There is only one square of side length $8$.

Rotating twice only fixes rectangles. There are $15$ such rectangles.

Rotating three times only fixes squares. There is only one square of side length
$8$.

Utilizing Burnside's Lemma, the number of unique quadrilaterals is $\frac{2255 +
1 + 15 + 1}{4} = \boxed{568}$.
\end{proof}
\fi

\begin{prb}[Catalan Numbers]
Consider a $n \times n$ grid. Compute the number of paths from $(0, 0)$ to $(n,
n)$ such that for every point $(x, y)$ on the path, $x \geq y$. Let this number
be $C_n$.

Show that $C_n$ is also the number of
\begin{enumerate}[(i)]
\item
$n$ pairs of parentheses being correctly matched.
\item
full binary trees (every node either has $0$ or $2$ children) with $n + 1$
leaves.
\item
ways to cut a convex polygon with $n + 2$ sides into triangles with non-crossing
line segments.
\item
binary trees with $n$ vertices.
\item
ways to tile a staircase of height $n$ with $n$ rectangles.
\end{enumerate}
\end{prb}

\ifsolutions
\begin{proof}[Solution]
There are clearly $\binom{2n}{n}$ paths without the condition. Consider the
first time an invalid path reaches above the diagonal, at some point $(x, x +
1)$. To get to $(n, n)$, it must take $n - x$ rightward steps and $n - x - 1$
upwards steps. We reflect the remaining path over the line $y = x + 1$. This new
path makes $n - x - 1$ rightward steps and $n - x$ upwards steps, and ends on
$(n - 1, n + 1)$. Notice that all bad paths must touch $y = x + 1$ and all paths
to $(n - 1, n + 1)$ also touch $y = x + 1$, and the same ``flipping'' process
can be performed to obtain a bad path. Hence this precisely enumerates the bad
paths, of which there are $\binom{2n}{n - 1}$ of. We then compute
\[ \begin{aligned}
C_n &= \binom{2n}{n} - \binom{2n}{n - 1} \\
&= \frac{(2n)!}{n! \cdot n!} - \frac{(2n)!}{(n - 1)! \cdot (n + 1)!} \\
&= \frac{(2n)!}{n! \cdot (n - 1)!} \left( \frac{1}{n} - \frac{1}{n + 1} \right)
\\
&= \frac{(2n)!}{n! \cdot (n - 1)!} \left( \frac{n + 1 - n}{n \cdot (n + 1)}
\right) \\
&= \frac{1}{n + 1} \binom{2n}{n}.
\end{aligned} \]

We then biject as follows.
\begin{enumerate}[(i)]
\item
Go right on left parentheses, go up on right parentheses.
\item
Consider an in order traversal of the tree (left, node, right). Every time we go
left, go right. Every time we go right, go up.
\item
Connect the centers of two triangles if they share an edge. This forms a binary
tree with $n$ vertices.
\item
Add a leaf to every node that does not already have a left and right subtree.
This becomes a full binary tree.
\item
We orient our staircase as follows. The first column has length $n$, the second
has $n - 1$, etc. Start at the top right ``block''. It belongs to a rectangle.
If the left edge touches a rectangle that has not been touched, connect the two.
If the bottom edge touches a rectangle, it must be untouched and connect the
two. This is a binary tree with $n$ vertices.
\end{enumerate}
\end{proof}
\fi

\begin{prb}
Determine the number of ways of filling a $2 \times n$ matrix with the numbers
$1, 2, \dots, 2n$ such that each row and each column has increasing entries.
\end{prb}

\ifsolutions
\begin{proof}[Solution]
We can always get a valid matrix by filling it with $1, 2, \dots, 2n$ in that
order in the following manner:

\begin{enumerate}[(i)]
\item
If the top row contains an equal number of filled entries as the bottom row,
fill the top.
\item
Otherwise, fill any row.
\end{enumerate}
This guarantees that each row and column is increasing and not following this
process guarantees some column is decreasing. This is easily seen as a bijection
to writing left parentheses when filling the top and writing right parentheses
for filling the bottom, yielding $n$ sets of correctly formatted parentheses,
which we know is $C_n = \frac{1}{n + 1} \binom{2n}{n}$.
\end{proof}
\fi

\begin{prb}[Vandermonde]
Prove that
\[ \binom{n + m}{k} = \sum_{i = 0}^k \binom{n}{i} \binom{m}{k - i}. \]
\end{prb}

\ifsolutions
\begin{proof}[Solution]
Examine coefficients of $(1 + x)^{m + n} = (1 + x)^m (1 + x)^n$. In particular,
look at the coefficient of $x^k$.
\end{proof}
\fi

\begin{prb}
Compute the sum
\[ \binom{2018}{0} + \binom{2018}{3} + \binom{2018}{6} + \cdots +
\binom{2018}{2016}. \]
\end{prb}

\ifsolutions
\begin{proof}[Solution]
Let $\omega$ be a primitive 3rd root of unity, so it satisfies $1 + \omega +
\omega^2 = 0$. Notice how
\[ (1 + \omega)^{2018} = \binom{2018}{0} + \binom{2018}{1} \omega +
\binom{2018}{2} \omega^2 + \binom{2018}{3} + \cdots + \binom{2018}{2016} +
\binom{2018}{2017} \omega + \binom{2018}{2018} \omega^2. \]
We can do the same with $\omega^2$.
\[ (1 + \omega^2)^{2018} = \binom{2018}{0} + \binom{2018}{1} \omega^2 +
\binom{2018}{2} \omega + \binom{2018}{3} + \cdots + \binom{2018}{2016} +
\binom{2018}{2017} \omega^2 + \binom{2018}{2018} \omega. \]
We can also do the same thing with $(1 + 1)^{2018}$ to get $1 + \omega +
\omega^2$ in the every slot that is not divisible by $3$ once we add everything
up. This is $3$ times the sum. Additionally, recall that $1 + \omega =
-\omega^2$ and $1 + \omega^2 = -\omega$. Thus, the value of the sum is
\[ \begin{aligned}
\frac{1}{3} \left( 2^{2018} + (-\omega^2)^{2018} + (-\omega)^{2018} \right) &=
\frac{1}{3} \left( 2^{2018} + \omega + \omega^2 \right) \\
&= \boxed{\frac{1}{3} \left(2^{2018} - 1\right)}.
\end{aligned} \]
\end{proof}
\fi

\begin{prb}[HMMT 2007 C-4]
On the Cartesian grid, Johnny wants to travel from $(0, 0)$ to $(5, 1)$, passing
through all twelve points of the rectangle $S = \set{(i, j) : 0 \leq i \leq 5, 0
\leq j \leq 1, i, j \in \ZZ}$. Each step, Johnny can go from one point in $S$ to
another in $S$ via a line segment connecting the two points. How many ways are
there for Johnny to start at $(0, 0)$ and end at $(5, 1)$ so that he never
crosses his own path?

\begin{center}
\begin{asy}
size(5cm);
for (int i = 0; i <= 5; ++i) {
  for (int j = 0; j <= 1; ++j) {
    dot((i, j));
  }
}

draw((0, 0) -- (0, 1) -- (1, 1) -- (2, 1) -- (3, 1) -- (1, 0) -- (2, 0) -- (3,
0) -- (4, 0) -- (4, 1) -- (5, 0) -- (5, 1));
\end{asy}
\end{center}
\end{prb}

\ifsolutions
\begin{proof}[Solution]
Notice that we must traverse $(0, 0), (1, 0), \dots, (5, 0)$ from left to right,
otherwise we will have an intersection. The same follows for $(0, 1), (1, 1),
\dots, (5, 1)$. Thus the number of paths is just $\binom{10}{5} = \boxed{252}$.
\end{proof}
\fi

\begin{prb}[2014 HMMT C-2]
There are $10$ people who want to choose a committee of $5$ people. They do this
by first selecting $1$, $2$, $3$, or $4$ committee leaders, who then choose
among the remaining people to complete the $5$-person commmittee. How many ways
can the committee be formed, assuming that people are distinguishable? (If two
committees have the same people but different leaders, then they are distinct.)
\end{prb}

\ifsolutions
\begin{proof}[Solution]
There are $\binom{10}{5} = 252$ ways of selecting $5$ people. We then need to
assign committee leaders to each selection. Each member is either a leader or
not, which yields $2^5$ possible selections, but they cannot all be leaders and
they cannot all be non-leaders, so we subtract $2$ to count the number of ways
leaders can be chosen. Then answer is then $30 \cdot 252 = \boxed{7560}$.
\end{proof}
\fi

\begin{prb}[2013 AIME II-9]
A $7 \times 1$ board is completely covered by $m \times 1$ tiles without
overlap; each tile may cover any number of consecutive squares, and each tile
lies completely on the board. Each tile is either red, blue, or green. Let $N$
be the number of tilings of the $7 \times 1$ board in which all three colors are
used at least once. For example, a $1 \times 1$ red tile followed by a $2 \times
1$ green tile, a $1 \times 1$ green tile, a $2 \times 1$ blue tile, and a $1
\times 1$ green tile is a valid tiling. Note that if the $2 \times 1$ blue tile
is replaced by two $1 \times 1$ blue tiles, this results in a different tiling.
Find the remainder when $N$ is divided by $1000$.
\end{prb}

\ifsolutions
\begin{proof}[Solution]
We list out how many ways we can tile the board with $n$ tiles, as well as how
many ways we can color them.

\begin{center}
\begin{tabular}{c | c | c}
$n$ & tilings & colorings \\ \hline
3 & 15 & 6 \\
4 & 20 & 36 \\
5 & 15 & 150 \\
6 & 6 & 540 \\
7 & 1 & 1806 \\
\end{tabular}
\end{center}

The number of ways to tile with $n$ tiles is simply $\binom{6}{n}$ and then the
number of colorings is PIE given by $3^n - 3 \cdot 2^n + 3$ with colorings with
$3$ colors, all $3$ colorings with $2$ colors, and the $3$ colorings with a
single color. Then compute, $15 \cdot 6 + 20 \cdot 36 + 15 \cdot 150 + 6 \cdot
540 + 1 \cdot 1806 = 8106$, so the answer is $\boxed{106}$.
\end{proof}
\fi

\begin{prb}
Determine the number of ways to stack coins in the plane such that the bottom
row consists of $n$ consecutive coins. For example, there are $5$ ways for the
case $n = 3$.

\begin{center}
\begin{asy}
size(10cm);

draw(circle((0, 0), 1));
draw(circle((2, 0), 1));
draw(circle((4, 0), 1));

draw(circle((8, 0), 1));
draw(circle((10, 0), 1));
draw(circle((12, 0), 1));
draw(circle((9, sqrt(3)), 1));

draw(circle((16, 0), 1));
draw(circle((18, 0), 1));
draw(circle((20, 0), 1));
draw(circle((19, sqrt(3)), 1));

draw(circle((24, 0), 1));
draw(circle((26, 0), 1));
draw(circle((28, 0), 1));
draw(circle((25, sqrt(3)), 1));
draw(circle((27, sqrt(3)), 1));

draw(circle((32, 0), 1));
draw(circle((34, 0), 1));
draw(circle((36, 0), 1));
draw(circle((33, sqrt(3)), 1));
draw(circle((35, sqrt(3)), 1));
draw(circle((34, 2 * sqrt(3)), 1));
\end{asy}
\end{center}
\end{prb}

\ifsolutions
\begin{proof}[Solution]
We turn every circle into a rotated square.

\begin{center}
\begin{asy}
size(10cm);

pen bb = blue + 1.5;

path sq(pair c, real r) {
  path p = (0, r) -- (r, 0) -- (0, -r) -- (-r, 0) -- cycle;
  return shift(c) * scale(r) * p;
}

draw(sq((0, 0), 1));
draw(sq((2, 0), 1));
draw(sq((4, 0), 1));
draw((-1, 0) -- (0, 1) -- (1, 0) -- (2, 1) -- (3, 0) -- (4, 1) -- (5,  0),
bb);

draw(sq((8, 0), 1));
draw(sq((10, 0), 1));
draw(sq((12, 0), 1));
draw(sq((9, 1), 1));
draw((7, 0) -- (8, 1) -- (9, 2) -- (10, 1) -- (11, 0) -- (12, 1) -- (13,  0),
bb);

draw(sq((16, 0), 1));
draw(sq((18, 0), 1));
draw(sq((20, 0), 1));
draw(sq((19, 1), 1));
draw((15, 0) -- (16, 1) -- (17, 0) -- (18, 1) -- (19, 2) -- (20, 1) -- (21,  0),
bb);

draw(sq((24, 0), 1));
draw(sq((26, 0), 1));
draw(sq((28, 0), 1));
draw(sq((25, 1), 1));
draw(sq((27, 1), 1));
draw((23, 0) -- (24, 1) -- (25, 2) -- (26, 1) -- (27, 2) -- (28, 1) -- (29,  0),
bb);

draw(sq((32, 0), 1));
draw(sq((34, 0), 1));
draw(sq((36, 0), 1));
draw(sq((33, 1), 1));
draw(sq((35, 1), 1));
draw(sq((34, 2), 1));
draw((31, 0) -- (32, 1) -- (33, 2) -- (34, 3) -- (35, 2) -- (36, 1) -- (37,  0),
bb);
\end{asy}
\end{center}
The answer is thus $C_n$.
\end{proof}
\fi

\begin{prb}[HMMT 2007 C-10]
A subset $S$ of the non-negative integers is called \emph{supported} if it
contains $0$, and for all $k \in S$, $k + 8, k + 9 \in S$. How many supported
sets are there?
\end{prb}

\ifsolutions
\begin{proof}[Solution]
One easily notes that $S \supseteq \set{0, 8, 9, 16, 17, 18, 24 \, \mh \, 27, 32
\, \mh \, 36, 40 \, \mh \, 45, 48 \, \mh \, 54, 56 \, \mh \, {+\infty}}$. We
organize the members of $S' = \ZZ_0^+ \setminus S$ as follows.

\begin{center}
\begin{asy}
size(10cm);

for (int i = 0; i <= 16; i += 2) {
  label("*", (i, 0));
}
for (int i = 1; i <= 15; i += 2) {
  label("*", (i, 1));
}
for (int j = 7; j >= 1; j -= 1) {
  for (int i = 1; i <= j; i += 1) {
    label(string(9 * (7 -  j) + i), (2 * i + 7 - j, 9 - j));
  }
}
\end{asy}
\end{center}

Notice that every path that starts at the bottom left ending at the bottom
right, moving only diagonally up and diagonally down, carves out a $S'$. Namely,
if we let $S$ contain, say $29$, then $S$ also contains $37, 38, 46, 47, 55$,
and this corresponds to the path
\[ 1 \to 10 \to 19 \to 28 \to 20 \to 12 \to 21 \to 30 \to 39 \to 31 \to 23 \to
15 \to 7. \]
Therefore the number of supported sets corresponds to the number of valid paths,
which is easily bijected to valid sets of parentheses, so the answer is
\[ C_8 = \frac{1}{9} \binom{16}{8} = \boxed{1430}. \]
\end{proof}
\fi

\begin{prb}[HMMT 2014 C-6]
We have a calculator with two buttons that displays an integer $x$. Pressing the
first button replaces $x$ with $\left\lfloor \frac{x}{2} \right\rfloor$, and
pressing the second button replaces $x$ with $4x + 1$. How many integers less
than or equal to $2014$ can be achieved through a sequence of arbitrary button
presses? (It is permitted for the number displayed to exceed $2014$ during the
sequence.)
\end{prb}

\ifsolutions
\begin{proof}[Solution]
Consider what the calculator does in binary. Pressing the first button truncates
the last digit and pressing the second button appends $01$ to the rightmost
side. Thus, the calculator is only able to display any binary string that does
not contain the substring $11$, as our insertion primitives consist of inserting
``0'' and inserting ``01''.

Let $F_n$ denote the number of valid strings of length $n$. If $x$ is a valid
string of length $n - 1$, then $x0$ is a valid string of length $n$. If $x$ is a
valid string of length $n - 2$, then $x01$ is a valid string of length $n$. This
clearly enumerates all valid strings, so we get the recurrence $F_n = F_{n - 1}
+ F_{n - 2}$ with $F_1 = 2$ and $F_2 = 3$. Notice that $2014 = 11111011110_2$,
which has $11$ digits, and the maximum valid string with $11$ digits is
$10101010101_2 = 1365$, so $F_{11}$ counts all of these and nothing more. Doing
some computation yields $F_{11} = \boxed{233}$.
\end{proof}
\fi

\begin{prb}[2001 AIME I-15]
The numbers $1$, $2$, $3$, $4$, $5$, $6$, $7$, and $8$ are randomly written on
the faces of a regular octahedron so that each face contains a different number.
The probability that no consecutive numbers (here we regard $8$ and $1$ to be
consecutive) are written on faces that share an edge is $\frac{m}{n}$, where $m$
and $n$ are relatively prime positive integers. Find $m + n$.
\end{prb}

\ifsolutions
\begin{proof}[Solution]
Consider the cube formed by joining the faces of the octahedron. The edges of
the cube represent the faces of the octahedron that share an edge.

\begin{center}
\begin{asy}
settings.render=0;
settings.prc=false;
size(6cm);
import three;
draw((0, 0, 0) -- (0, 0, 1) -- (0, 1, 1) -- (1, 1, 1) -- (1, 0, 1) -- (1, 0, 0)
-- (1, 1, 0) -- (0, 1, 0) -- (0, 0, 0));
draw((1, 0, 1) -- (0, 0, 1));
draw((1, 0, 0) -- (0, 0, 0));
draw((1, 1, 0) -- (1, 1, 1));
draw((0, 1, 0) -- (0, 1, 1));
for (int i = 0; i < 2; ++i) {
  for (int j = 0; j < 2; ++j) {
    for (int k = 0; k < 2; ++k) {
      dot((i, j, k));
    }
  }
}
\end{asy}
\end{center}

An assignment of each number to each vertex corresponds to a Hamiltonian cycle
on the graph of diagonals of this cube, by walking the paths defined by $1 \to
2, 2 \to 3$, etc. We are interested in the number of these graphs. Notice that
the diagonals of length $\sqrt{2}$ form two tetrahedrons with no common vertex
and the diagonals of length $\sqrt{3}$ join the vertices of each tetrahedron. We
can then ``flatten'' the graphs of the two tetrahedrons into squares and connect
the vertices like so.

\begin{center}
\begin{asy}
settings.render=0;
settings.prc=false;
size(6cm);
import three;
draw((0, 0, 0) -- (0, 0, 1) -- (0, 1, 1) -- (1, 1, 1) -- (1, 0, 1) -- (1, 0, 0)
-- (1, 1, 0) -- (0, 1, 0) -- (0, 0, 0));
draw((1, 0, 1) -- (0, 0, 1));
draw((1, 0, 0) -- (0, 0, 0));
draw((1, 1, 0) -- (1, 1, 1));
draw((0, 1, 0) -- (0, 1, 1));
draw((0, 0, 0) -- (1, 1, 0));
draw((0, 1, 0) -- (1, 0, 0));
draw((0, 0, 1) -- (1, 1, 1));
draw((0, 1, 1) -- (1, 0, 1));
for (int i = 0; i < 2; ++i) {
  for (int j = 0; j < 2; ++j) {
    for (int k = 0; k < 2; ++k) {
      dot((i, j, k));
    }
  }
}
\end{asy}
\end{center}

In order to complete a Hamiltonian cycle, we must use the vertical edges either
two or four times.

To form a path using four vertical edges, we must alternate between vertical and
flat edges. We first pick two vertices from the top square and draw their edges
downwards. From there, we can visit the other two unvisited vertices in $2$
ways. Then we must draw edges back up, and connect back to our original vertices
in $1$ way (the other way forms two squares). There are $\binom{4}{2} = 6$ ways
of picking the first two vertices, and each of these ``octagons'' correspond to
$8 \cdot 2$ different labellings of the dice. There are then $8 \cdot 6 \cdot 2
= 96$ such octagons.

To form a path using two vertical edges, we have two sets of $1$ vertical and
$3$ flat edges. Pick two vertices on the top square and draw their vertical
edges. For each of the top and bottom squares, there are $2$ different ways of
completing the path that starts at one vertex and ends at the other. Each of
these octagons also correspond to $8 \cdot 2$ different labellings. Thus, there
are $8 \cdot 6 \cdot 2 \cdot 2 \cdot 2 = 384$ octagons formed in this manner.

There are $8! = 40320$ different labellings of the dice, so the probability is
$\frac{96 + 384}{40320} = \frac{480}{40320} = \frac{1}{84}$, so the answer is $1
+ 84 = \boxed{085}$.
\end{proof}
\fi
